\section{Introduction}
%Institute of Field Robotics(FIBO) at King Mongkut's University of Technology Thonburi (KMUTT) has developed the humanoid robots to participate in RoboCup humanoid kid-sized league since 2005 under the name 'Team KMUTT' and later on 'Pheonix'. The successor, 'Hanuman-FC', won the Thailand Humanoid Soccer Robot Championship 2012 and participated in World-RoboCup 2013 (Eindhoven, Netherlands) under the name 'Hanuman-KMUTT'. In World-RoboCup 2013, our team passed into the quarter-final round. At the end of 2013, Our team won the Thailand Humanoid Soccer Robot Championship 2013. And then we joined the World-RoboCup 2014 (João Pessoa, Brazil) and passed into round robin 2. Because of limited funding and Thailand has no longer host the local competition in the country, our humanoid team development has been suspended. Last year, Robocup Asia-Pacific 2017 were host in Thailand, We participated after 3 years break with new undergraduate members. This year ...
%\paragraph{}
%In this paper, we will describe the recent development of our kid-sized humanoid robots. Section 2 gives an overview of the system design in our robots which consists of two strikers and one goalie. In section 3, the vision based navigation system of the robot will be explained. Section 4 discusses the game control and decision-making system. The last section concludes the paper.

Institute of Field Robotics(FIBO) at King Mongkut's University of Technology Thonburi (KMUTT) has developed the humanoid robot for paticipate humanoid kid-sized league under the name 'Team KMUTT'. In 2012, Our team under name 'Hanuman-FC' won the Thailand Humanoid Soccer Robot Championship. We participated World-Robocup 2013 (Eindhoven, Netherlands), and 2014 (João Pessoa, Brazil). At World-Robocup 2013, we passed into quarter-final. Our humanoid team development has been suspended since Thailand had no longer host the local league. In 2017, Thailand hosted Robocup Asia-Pacific 2017. It's our chance to participate humanoid league after 3 years break. Many new undergraduate member alongside some former members managed to win the 3rd place at the Robocup Asia-Pacific 2017. Our major development this year focuses on the new software framework and the new vision system. We expect to improve on the localization capability which our team did not have much success in the past. This year, we are looking forward to participate in Robocup 2019 (Sydney, Australia).